% LaTeX Curriculum Vitae -- Modified by Brad Duthie
%
% Copyright (C) 2004-2009 Jason Blevins <jrblevin@sdf.lonestar.org>
% http://jblevins.org/projects/cv-template/
%
% You may use use this document as a template to create your own CV
% and you may redistribute the source code freely. No attribution is
% required in any resulting documents. I do ask that you please leave
% this notice and the above URL in the source code if you choose to
% redistribute this file.

\documentclass[letterpaper]{article}

\usepackage{fancyhdr}
\usepackage{hyperref}
\usepackage{geometry}
\usepackage{lipsum}
\usepackage{bbding}
\usepackage{ifsym}
\usepackage{marvosym}
\usepackage{url}
\usepackage{xcolor}
\usepackage{graphicx}
\usepackage{etaremune}

% Comment the following lines to use the default Computer Modern font
% instead of the Palatino font provided by the mathpazo package.
% Remove the 'osf' bit if you don't like the old style figures.
\usepackage[T1]{fontenc}
\usepackage[sc,osf]{mathpazo}

% Set your name here
\def\name{A. Bradley Duthie}

% Replace this with a link to your CV if you like, or set it empty
% (as in \def\footerlink{}) to remove the link in the footer:
\def\footerlink{http://bradduthie.github.io/DuthieCV.pdf}

%\newcommand*\myglobe{%
% as suggested by Martin ;-)
%\includegraphics[height=1.6ex]{globe.pdf}}

%\fancypagestyle{specialfooter}{%
%  \fancyhf{}
%  \renewcommand\headrulewidth{0pt}
%  \fancyfoot[L]{Email: aduthie@abdn.ac.uk, brad.duthie@gmail.com 
%  }
%}

% The following metadata will show up in the PDF properties
\hypersetup{
  colorlinks = true,
  urlcolor = black,
  pdfauthor = {\name},
  pdfkeywords = {evolution, ecology, coexistence},
  pdftitle = {\name: Curriculum Vitae},
  pdfsubject = {Curriculum Vitae},
  pdfpagemode = UseNone
}

\geometry{
  body={6.5in, 8.5in},
  left=1.0in,
  top=1.25in
}

% Customize page headers
\pagestyle{myheadings}
\markright{\name}
\thispagestyle{empty}

% Custom section fonts
\usepackage{sectsty}
\sectionfont{\rmfamily\mdseries\Large}
\subsectionfont{\rmfamily\mdseries\itshape\large}

% Other possible font commands include:
% \ttfamily for teletype,
% \sffamily for sans serif,
% \bfseries for bold,
% \scshape for small caps,
% \normalsize, \large, \Large, \LARGE sizes.

% Don't indent paragraphs.
\setlength\parindent{0em}

% Make lists without bullets
\renewenvironment{itemize}{
  \begin{list}{}{
    \setlength{\leftmargin}{1.5em}
  }
}{
  \end{list}
}

\begin{document}

\thispagestyle{specialfooter}
% Place name at left


{\Huge \name}

\hrulefill

\vspace{0.03in}

%I am a comparative evolutionary biologist interested in the evolution of phenotypic diversity from a theoretical and empirical perspective. I study microevolutionary and macroevolutionary trends in phenotypic diversification to elucidate the historical and contemporary forces responsible for them. I focus on a multivariate characterization of the phenotype (morphometrics), and the inherent complexities of the evolutionary process in multi-dimensional phenotype space. My empirical work examines these patterns and processes in animals; primarily plethodontid salamanders. In my theoretical work, I develop new analytical, statistical, and theoretical tools to address fundamental questions in evolutionary biology. 

I am an evolutionary ecologist interested in the interactions among species and individuals within complex communities and populations. My theoretical research uses mathematical and computational modelling (primarily individual-based models) to understand coexistence among multiple interacting species, and the evolution of inbreeding strategy and polyandry in populations characterised by multiple interacting individuals. My empirical work primarily uses insect communities to test fundamental questions about mechanisms underlying species distributions and coexistence. %As an educator, I take a learning-centred approach to guide students through course material, and I evaluate my teaching as successful when I have evidence that students are thinking actively and effectively about biological concepts.

% Orcid ID (work in somewhere?):  0000-0001-8343-4995 
% Scopus Author ID:               55596393800


\hrulefill
% Alternatively, print name centered and bold:
%\centerline{\huge \bf \name}

\vspace{0.16in}

%{\small School of Biological Sciences, Zoology Building, University of Aberdeen, Tillydrone Avenue, Aberdeen, AB24 2TZ} \\
%{\small Phone: (+44) 012242 73255 \t Email: \href{mailto:aduthie@abdn.ac.uk}{\tt aduthie@abdn.ac.uk}, \href{mailto:brad.duthie@gmail.com}{\tt brad.duthie@gmail.com} \t  Homepage: & \href{http://db.tt/nlQfxTvc}{\tt http://db.tt/nlQfxTvc}} \\

\begin{minipage}{0.65\linewidth}
\section*{Professional Appointments}
\begin{itemize}
	\item{\bf Leverhulme Trust Early Career Fellow}, Nov. 2017 -- Oct. 2020 \\
	Biological and Environmental Sciences, University of Stirling
\end{itemize}
\begin{itemize}
	\item{\bf ERC Postdoctoral Research Fellow}, Nov. 2016 -- Oct. 2017 \\
	Biological and Environmental Sciences, University of Stirling
\end{itemize}
\begin{itemize}
	\item{\bf ERC Postdoctoral Research Fellow}, Jun. 2013 -- Oct. 2016 \\
	School of Biological Sciences, University of Aberdeen
\end{itemize}

\section*{Education}
\begin{itemize}
  \item {\bf Ph.D. Ecology and Evolutionary Biology}, Spring 2013 \\
	Graduate Minor: Statistics \\
	Certificate: Graduate Student Teaching Certificate \\
	Iowa State University
  \item {\bf B.S. Biology} (Magna cum laude), Spring 2007 \\
	Specialisation: Ecology, Evolution, Environment \\
	Southern Illinois University Edwardsville
  \item {\bf B.S. Philosophy} (Magna cum laude), Spring 2007 \\
	Minor: Chemistry \\
	Southern Illinois University Edwardsville
\end{itemize}


\end{minipage}
\begin{minipage}{0.35\linewidth}
\fcolorbox{black}{grey!10}{
 {\small
  \begin{tabular}{ll}
  \\ 
 {\bf
 A. Bradley Duthie} \\{\bf
 Ph.D., FHEA} \\
 Evolutionary Ecologist \\
 Ecological Modeller \\
 Biostatistician \\
 \\
 3B156 \\
 University of Stirling \\
 Stirling \\
 FK9 4LA \\
 United Kingdom \\ \\ \\
 \setlength\tabcolsep{1.65pt}
 \begin{tabular}{ll}
 \Phone & (+44) 01786 467787 \\
 \Mobilefone & (+44) 07561 408101 \\
 \Envelope & \href{mailto:alexander.duthie@stir.ac.uk}{alexander.duthie@stir.ac.uk} \\
 & \href{mailto:brad.duthie@gmail.com}{brad.duthie@gmail.com} \\
 \Mundus~ & \href{http://bradduthie.github.io}{http://bradduthie.github.io}\\ 
 \includegraphics[scale=0.08]{GitHub.pdf} & \href{https://github.com/bradduthie}{bradduthie}\\
 %\aiOrcidSquare & 0000-0001-8343-4995 
 \end{tabular}\\ \\
 \end{tabular}
 }
}
\end{minipage}


\section*{Publications}
\begin{etaremune}
\item {\bf Duthie, A. B.}, J. J. Cusack, I. L. Jones, E. B. Nilsen, R. A. Pozo, O. S. Rakotonarivo, B. Van Moorter, and N. Bunnefeld. GMSE: an R package for generalised management strategy evaluation. {\it Methods in Ecology and Evolution}. {\it Submitted}. \href{https://www.biorxiv.org/content/early/2017/11/17/221432}{DOI: 10.1101/221432} %{\it Methods in Ecology and Evolution}. 
%\item Redpath, S. M., Andren, Z. Baynham-Herd, N. Bunnefeld, {\bf A. B. Duthie}, Frank, C. A. Garcia, A. Keane, J. M\r{a}nsson, L. Nilsson, C. R. J. Pollard, O. S. Rakotonarivo, C. F. Salk, and H. Travers. Games as tools to address conservation conflicts. {\it Trends in Ecology and Evolution}. {\it In prep}.
%\item Cusack, J. J., {\bf A. B. Duthie}, O. S. Rakotonarivo, R. A. Pozo, T. H. E. Mason, J. M\r{a}nsson, L. Nilsson, R. McKenzie, I. M. Tombre, E. Eyth\'{o}rsson, J. Madsen, A. Tulloch, G. Churchill, J. Shaw, R. D. Hearn, S. Redpath, and N. Bunnefeld. Short and long-term synchrony between changes in conflict management effort and increasing large grazing bird populations in northern Europe. {\it Conservation Letters}. {\it In Review}
%\item Paine, C. E. T., A. Deasey, and {\bf A. B. Duthie}. Towards general mechanistic predictions of community dynamics. {\it Functional Ecology}. {\it In review}.
\item {\bf Duthie, A. B.}, G. Bocedi, R. R. Germain, and J. M. Reid. 2017. Evolution of pre-copulatory and post-copulatory strategies of inbreeding avoidance and associated polyandry. {\it Journal of Evolutionary Biology}. {\it In press}. \href{https://onlinelibrary.wiley.com/doi/10.1111/jeb.13189/full}{DOI: 10.1111/jeb.13189}
\item {\bf Duthie, A. B.}, A. M. Lee, and J. M. Reid. 2016. Inbreeding parents should invest more resources in fewer offspring. {\it Proceedings of the Royal Society B}. 282:20161845. \href{http://rspb.royalsocietypublishing.org/content/283/1843/20161845}{DOI: 10.1098/rspb.2016.1845}
\item {\bf Duthie, A. B.}, and J. M. Reid. 2016. Evolution of inbreeding avoidance and inbreeding preference through mate choice among interacting relatives. {\it American Naturalist}. 188 (6):651-667. \href{http://www.journals.uchicago.edu/doi/full/10.1086/688919}{DOI: 10.1086/688919}
\item {\bf Duthie, A. B.}, and J. D. Nason. 2016. Plant connectivity underlies plant-pollinator-exploiter distributions in {\it Ficus petiolaris} and associated pollinating and non-pollinating fig wasps. {\it Oikos}. 125 (11):1597-1606. \href{http://onlinelibrary.wiley.com/doi/10.1111/oik.02905/abstract}{DOI: 10.1111/oik.02905}
\item {\bf Duthie, A. B.}, G. Bocedi, and J. M. Reid. 2016. When does female multiple mating evolve to adjust inbreeding? Effects of inbreeding depression, direct costs, mating constraints, and polyandry as a threshold trait. {\it Evolution}. 70 (9):1927-1943. \href{http://onlinelibrary.wiley.com/doi/10.1111/evo.13005/abstract}{DOI: 10.1111/evo.13005}
\item Frater, P. N., and {\bf A. B. Duthie}. 2016. Power scaling, vascular branching patterns, and the golden ratio. {\it Ideas in Ecology and Evolution}. 9:15-18. DOI: \href{http://ojs.library.queensu.ca/index.php/IEE/article/view/6312}{10.4033/iee.2016.9.4.n}
\item Reid, J. M., G. Bocedi, P. Nietlisbach, {\bf A. B. Duthie}, M. E. Wolak, E. A. Gow, and P. Arcese. 2016. Variation in parent-offspring kinship in socially monogamous systems with extra-pair reproduction and inbreeding. {\it Evolution}. 70 (7):1512-1529. \href{http://onlinelibrary.wiley.com/doi/10.1111/evo.12953/abstract}{DOI: 10.1111/evo.12953}
\item Reid, J. M., P. Arcese, G. Bocedi, {\bf A. B. Duthie}, M. E. Wolak, and L. F. Keller. 2015. Resolving the conundrum of inbreeding depression but no inbreeding avoidance: estimating sex-specific selection on inbreeding by song sparrows ({\it Melospiza melodia}). {\it Evolution}. 69 (11):2846-2861. \href{http://onlinelibrary.wiley.com/doi/10.1111/evo.12780/abstract}{DOI: 10.1111/evo.12780}
\item {\bf Duthie, A. B.}, K. C. Abbott, and J. D. Nason. 2015. Trade-offs and coexistence in fluctuating environments: evidence for a key dispersal-fecundity trade-off in five nonpollinating fig wasps. {\it American Naturalist}. 186 (1):151-158. \href{http://www.jstor.org/stable/10.1086/681621}{DOI: 10.1086/681621}
\item Reid, J. M., {\bf A. B. Duthie}, M. E. Wolak, and P. Arcese. 2015. Demographic mechanisms of inbreeding adjustment through extra-pair reproduction. {\it Journal of Animal Ecology}. 84 (4):1029-1040. \href{http://onlinelibrary.wiley.com/doi/10.1111/1365-2656.12340/abstract}{DOI: 10.1111/1365-2656.12340}
\item {\bf Duthie, A. B.}, and J. M. Reid. 2015. What happens after inbreeding avoidance? Inbreeding by rejected relatives and the inclusive fitness benefit of inbreeding avoidance. {\it PLoS One}. 10 (4):e0125140. \href{http://journals.plos.org/plosone/article?id=10.1371/journal.pone.0125140}{DOI: 10.1371/journal.pone.0125140}
\item Duthie, A. C., and {\bf A. B. Duthie}. 2015. Do music and art influence one another? Measuring cross-modal similarities in music and art. {\it Polymath: An Interdisciplinary Arts and Sciences Journal}. \href{https://ojcs.siue.edu/ojs/index.php/polymath/article/view/3013}{5 (1):1-22}.
\item Reid, J. M., P. Arcese, L. F. Keller, R. R. Germain, {\bf A. B. Duthie}, S. Losdat, M. E. Wolak, and P. Nietlisbach. 2015. Quantifying inbreeding avoidance through extra-pair reproduction. {\it Evolution}. 69 (1):59-74. \href{http://onlinelibrary.wiley.com/doi/10.1111/evo.12557/abstract}{DOI: 10.1111/evo.12557}
\item {\bf Duthie, A. B.}, K. C. Abbott, and J. D. Nason. 2014. Trade-offs and coexistence: A lottery model applied to fig wasp communities. {\it American Naturalist}. 183 (6):826-841. \href{http://www.jstor.org/stable/10.1086/675897}{DOI: 10.1086/675897}
\item {\bf Duthie, A. B.}, and M. R. Falcy. 2013. The influence of habitat autocorrelation on plants and their seed-eating pollinators. {\it Ecological Modelling}. 251:260-270. \href{http://www.sciencedirect.com/science/article/pii/S0304380013000021}{DOI: 10.1016/j.ecolmodel.2012.12.019}
\item {\bf Duthie, A. B.} 2004. The fork and the paperclip: A memetic perspective. {\it Journal of Memetics - Evolutionary Models of Information Transmission}. 8 (1). \href{http://cfpm.org/jom-emit/2004/vol8/duthie_ab.html}{http://cfpm.org/jom-emit/2004/vol8/duthie\_ab.html}.
\end{etaremune}

\section*{Grant and Awards}
\begin{itemize}
\item Leverhulme Trust Early Career Fellowship (2017) -- 88,650 GBP
\item NSF Postdoctoral Research Fellowship in Biology (2013; award declined) -- 138,000 USD
\item NSF Doctoral Dissertation Improvement Grant (2010) -- 13,411 USD
\item Iowa State University Teaching Excellence Award (top 10\% of TAs; 2010)
\item Biotechnology Graduate Fellowship -- 20,000 USD (ISU; 2007)
\item Donal G. Myer Outstanding Student Award (SIUE; awarded to one science student; 2007)
\item Ober Honours Award in Philosophy (SIUE; awarded to one philosophy student; 2007)
\item Phi Kappa Phi (Collegiate Honour Society; 2006)
\item Chancellor's Scholarship (SIUE; full tuition for undergraduate education; 2003-2007)
\end{itemize}

%\newpage

\section*{Teaching Experience}
\hrulefill

%\vspace{0.06in}

As an educator, I take a learning-centred approach to guide students through course material, and I evaluate my teaching as successful when I have evidence that students are thinking actively and effectively about biological concepts. I have experience developing and delivering course content, and in student assessment, in both traditional and online formats. I have additionally developed \href{https://bradduthie.shinyapps.io/EcoEdu/}{learning software}, specifically for teaching numerical and individual-based ecological modelling to undergraduate students.

\hrulefill

%\vspace{0.03in}

\begin{itemize}
\item {\bf Professional Development:}
\begin{itemize}
\item[$\bullet$]{{\bf Fellow of the Higher Education Academy}}
\item[$\bullet$]{Principles of Learning and Teaching in Higher Education programme. University of Aberdeen.}
\item[$\bullet$]{Preparing Future Faculty (PFF) Program: Iowa State University. PFF Associate.}
\item[$\bullet$]{Graduate Student Teaching Certificate (GSTC) Program (PFF Track): Iowa State University.}
\end{itemize}

\item {\bf University Instruction:}
\begin{itemize}
\item[$\bullet$]{{\it Biology Field Course -- Course Contributor: statistics lecture} (SCIU3FB; Undergraduate: 20 credits) University of Stirling. Fall 2017.}
\item[$\bullet$]{{\it Introduction to Ecological and Environmental Modelling -- Course Contributor} (BI 4803; Undergraduate: 15 credits). University of Aberdeen. Spring 2015, 2016.}
\item[$\bullet$]{{\it Biology for Undergraduates (BUGS) -- Tutorial Session ZOO-4} (BI 1006; Undergraduate: 7.5 credits). University of Aberdeen. Fall 2014.}
\item[$\bullet$]{{\it Biological Evolution} (BIOL 315; Undergraduate: 3 credits). Iowa State University. Summer 2011, 2012*, Spring 2013*. *Distance Learning Course (online).}
\end{itemize}

\item {\bf University Teaching Assistantships:}
\begin{itemize}
\item[$\bullet$]{{\it Biological Evolution} (BIOL 315; Undergraduate: 3 credits). Iowa State University. Spring 2011-2012.}
\item[$\bullet$]{{\it Principles of Biology Laboratory I} (BIOL 211L; Undergraduate: 1 credit). Iowa State University. Fall 2008, 2012.}
\item[$\bullet$]{{\it Principles of Biology Laboratory II} (BIOL 212L; Undergraduate: 1 credit). Iowa State University. Fall 2009-2011, Spring 2008-2010.}
\end{itemize}

\end{itemize}

\section*{Student Mentoring}
\begin{itemize}
\item {\bf Mentored undergraduate researchers at Iowa State University}. Cody Duncan (2010), Megan Stoner (2010-2011), Elizabet Howe (2011), Christina Pacholec (2011), Ethan Van Helten (2011), Christine Lim (2011-2012), and Aisha Azhar (2012-2013).
\end{itemize}

\section*{Leadership}
\begin{itemize}
\item Manuscript reviewer for {\it Acta Oecologica}, {\it American Naturalist}, {\it Arthropod-Plant Interactions}, {\it Behavioral Ecology and Sociobiology}, {\it Biology Letters}, {\it BMC Evolutionary Biology}, {\it Evolution} (2), the {\it Journal of Animal Ecology}, the {\it Journal of Applied Ecology} (2), the {\it Journal of Theoretical Biology} (3), {\it Landscape Ecology}, {\it Oikos}, {\it Proceedings of the Royal Society B}, and {\it Symbiosis}
\item Grant reviewer for the National Science Foundation (USA) 
\item EEOB Senator -- Graduate and Professional Student Senate (ISU; 2011-2013)
\item Organiser -- Midwest Ecology and Evolution Conference at Iowa State University (2010)
\item Vice President -- Graduate Research in Evolutionary Biology and Ecology (ISU; 2009-2010)
\end{itemize}

\section*{Computer Skills}
\begin{itemize}
\item Experience in statistical analysis, mathematical \& individual-based modelling, genetic algorithms, and computer programming in C, R, and MATLAB
\item Use of Linux, Git, and high performance computing clusters
\item Knowledge of HTML, Markdown, \LaTeX, and MS Office Suite
\item Course design and instruction in Blackboard
%\item Course instruction in WebCT
\end{itemize}

\section*{Software Developed}

All listed software is publicly available on the \href{https://cran.r-project.org/}{Comprehensive R Archive Network} (CRAN) and on \href{https://github.com/bradduthie}{GitHub}.

\begin{etaremune}
\item {\bf GMSE} {\it v0.3.1.9}. 2017. Generalised Management Strategy Evaluation Simulator. {\it Written in R and C.} \\
R package link: \href{http://CRAN.R-project.org/package=GMSE}{< http://CRAN.R-project.org/package=GMSE >}
\item {\bf gamesGA} {\it v1.1.3.2}. 2017. Genetic Algorithm for Sequential Symmetric Games. {\it Written in R and C.} \\
R package link: \href{http://CRAN.R-project.org/package=gamesGA}{< http://CRAN.R-project.org/package=gamesGA >}
\end{etaremune}

\section*{Presentations}
\begin{itemize}
\item $\ddagger$Denotes invited talk, *Denotes undergraduate author
\item $\ddagger${\bf Duthie, A. B.} 2017. GMSE: a general tool for management strategy evaluation. {\bf Mathematics and Statistics Group Seminar} at the University of Stirling, Stirling, Scotland.
\item $\ddagger${\bf Duthie, A. B.} 2017. \href{https://bradduthie.shinyapps.io/GRIMSO}{Introduction to genetic algorithms, and their potential to link complex games to real-world behaviour in conservation conflicts}. {\bf Workshop: ``Behavioural games and conflict''} at Grims\"{o} Wildlife Research Station, Grims\"{o}, Sweden.
\item {\bf Duthie, A. B.}, G. Bocedi, and J. M. Reid. 2015. Evolution of polyandry and inbreeding avoidance: a genetically-explicit finite population model. {\bf Population Genetics Group Meeting 49} at the University of Edinburgh, Edinburgh, Scotland.
\item {\bf Duthie, A. B.}, G. Bocedi, and J. M. Reid. 2015. Should females mate multiply to avoid inbreeding? Insights from a computational modelling approach. {\bf The Scottish Informatics and Computer Science Alliance (SICSA) workshop on Computational Ecology} at the University of Edinburgh, Edinburgh, Scotland.
\item {\bf Duthie, A. B.} and J. M. Reid. 2014. Evolution of inbreeding and inbreeding avoidance among multiple relatives. {\bf Workshop: ``Evolution of Mating Systems''} at the Konnevesi Research Station of the University of Jyv\"{a}skyl\"{a}, Konnevesi, Finland.
\item {\bf Duthie, A. B.}, Lim*, C., and J. D. Nason. 2013. The effects of life history characteristics on male dimorphism frequency in two species of non-pollinating fig wasps. {\bf Ento '13 International Symposium and Annual National Science Meeting} at the University of Saint Andrews, Saint Andrews, Scotland.
\item {\bf Duthie, A. B.}, K. C. Abbott, and J. D. Nason. 2013. \href{https://www.youtube.com/watch?v=8Oi48FdaLXY}{A fluctuating environment drives coexistence in five non-pollinating fig wasps}. {\bf XIV Conference of the European Society for Evolutionary Biology} at the University of Lisbon, Lisbon, Portugal.
\item Lim*, C., {\bf A. B. Duthie}, and J. D. Nason. 2012. Violence inside the fig: Sexual selection among two species of non-pollinating male fig wasps. {\bf Symposium on Undergraduate Research and Creative Expression} at Iowa State University, Ames, Iowa.
\item {\bf Duthie, A. B.}, K. C. Abbott, and J. D. Nason. 2012. Does the storage effect maintain coexistence in non-pollinating fig wasp communities? {\bf Nineteenth Annual GREBE Spring Symposium} at Iowa State University, Ames, Iowa.
\item {\bf Duthie, A. B.} 2010. Turn-taking in habitat patch selection can be evolutionarily stable. {\bf Midwest Ecology and Evolution Conference} at Iowa State University, Ames, Iowa.
\item Day, K., {\bf A. B. Duthie}, and J. D. Nason. 2009. Cryptic parasites and their effects on the fitness and evolutionary stability of the fig-pollinator mutualism. {\bf Evolution} at the University of Idaho, Moscow, Idaho.
\item {\bf Duthie, A. B.} and M. Falcy. 2009. Spatial heterogeneity promotes coexistence in a mutualist/exploiter community. {\bf Sixteenth Annual GREBE Spring Symposium} at Iowa State University, Ames, Iowa.
\item {\bf Duthie, A. B.} and J. D. Nason. 2008. Dynamics of a Sonoran fig wasp community: A graphical modeling approach. {\bf Evolution} at the University of Minnesota, Minneapolis, Minnesota.
\item {\bf Duthie, A. B.} and J. D. Nason. 2008. Graphical modeling and the community dynamics of a sonoran desert fig, its pollinator, and exploiter wasps. {\bf Fifteenth Annual GREBE Spring Symposium} at Iowa State University, Ames, Iowa.
\item {\bf Duthie, A. B.} and M. Smith. 2007. Environmental effects on achenes of {\it Boltonia decurrens}, a threatened floodplain species: Age, mass, germination, and viability. {\bf Illinois State Academy of Sciences Annual Meeting} at the Illinois State Museum, Springfield, Illinois.
\item {\bf Duthie, A. B.}. 2006. This brain owns itself: How things like us project a unified agent {\bf First Annual Undergraduate Philosophy Conference} at Southern Illinois University Edwardsville, Edwardsville, Illinois.
\end{itemize}

\section*{Posters}
\begin{itemize}
\item {\bf Duthie, A. B.}, and J. M. Reid. 2015. The effect of relatedness structure and sexual conflict on the evolution of inbreeding avoidance and preference {\bf XV Conference of the European Society for Evolutionary Biology} at Lausanne, Switzerland.
\item {\bf Duthie, A. B.}, K. C. Abbott, and J. D. Nason. 2012. Does the storage effect maintain coexistence in non-pollinating fig wasp communities? {\bf Evolution} at Ottawa, Canada.
\item {\bf Duthie, A. B.}, K. C. Abbott, and J. D. Nason. 2011. Coexistence in fig wasp communities: A lottery model. {\bf Workshop 5: Coevolution and the ecological structure of plant-insect communities} at the Mathematical Biosciences Institute in Columbus, Ohio.
\item {\bf Duthie, A. B.}, and M. R. Falcy. 2011. Spatial heterogeneity promotes coexistence in a mutualist/exploiter community. {\bf Workshop 5: Coevolution and the ecological structure of plant-insect communities} at the Mathematical Biosciences Institute in Columbus, Ohio.
\end{itemize}

\section*{Professional Affiliations}
\begin{itemize}
\item American Society of Naturalists%, European Society for Evolutionary Biology
\item European Society for Evolutionary Biology
\end{itemize}

\section*{References}

\begin{minipage}{0.5\linewidth}

\begin{itemize}
\item {\bf Prof. Jane Reid} \\
School of Biological Sciences \\
Zoology Building \\
University of Aberdeen \\
Aberdeen, UK, AB24 2TZ \\
Phone: +44 01224 274224 \\
Email: jane.reid@abdn.ac.uk \\

\item {\bf Prof. Nils Bunnefeld} \\
Department Biological and \\ 
Environmental Sciences \\
University of Stirling \\
3B164, Stirling, UK, FK9 4LA \\
Phone: +44 1786 467804 \\
Email: nils.bunnefeld@stir.ac.uk \\

\end{itemize}
\end{minipage}
\begin{minipage}{0.5\linewidth}

\begin{itemize}

\newline
\item {\bf Prof. John Nason} \\
Department of Ecology, Evolution, \\ and Organismal Biology \\
339 Bessey Hall \\
Iowa State University \\
Ames, Iowa, USA 50011 \\
Phone: 1-515-294-2268 \\
Email: jnason@iastate.edu

\item {\bf Prof. Justin Travis} \\
School of Biological Sciences \\
Zoology Building \\
University of Aberdeen \\
Aberdeen, UK, AB24 2TZ \\
Phone: +44 01224 274483 \\
Email: justin.travis@abdn.ac.uk \\

%\item {\bf Prof. Brent Danielson} \\
%Department of Ecology, Evolution, \\ and Organismal Biology \\
%253 Bessey Hall \\
%Iowa State University \\
%Ames, Iowa, USA 50011 \\
%Phone: 1-515-294-5248 \\
%Email: brentd@iastate.edu


\end{itemize}

\end{minipage}

\bigskip

% Footer
%\begin{center}
%  \begin{footnotesize}
%    Last updated: \today \\
%    \href{\footerlink}{\texttt{\footerlink}}
%  \end{footnotesize}
%\end{center}

\end{document}
