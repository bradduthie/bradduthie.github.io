% LaTeX Curriculum Vitae -- Modified by Brad Duthie
%
% Copyright (C) 2004-2009 Jason Blevins <jrblevin@sdf.lonestar.org>
% http://jblevins.org/projects/cv-template/
%
% You may use use this document as a template to create your own CV
% and you may redistribute the source code freely. No attribution is
% required in any resulting documents. I do ask that you please leave
% this notice and the above URL in the source code if you choose to
% redistribute this file.

\documentclass[letterpaper]{article}

\usepackage{fancyhdr}
\usepackage{hyperref}
\usepackage{geometry}
\usepackage{lipsum}
\usepackage{bbding}
\usepackage{ifsym}
\usepackage{marvosym}
\usepackage{url}
\usepackage{xcolor}
\usepackage{graphicx}
\usepackage{etaremune}
\usepackage{float}
\usepackage{longtable}

% Comment the following lines to use the default Computer Modern font
% instead of the Palatino font provided by the mathpazo package.
% Remove the 'osf' bit if you don't like the old style figures.
\usepackage[T1]{fontenc}
\usepackage[sc,osf]{mathpazo}

% Set your name here
\def\name{A. Bradley Duthie}

% Replace this with a link to your CV if you like, or set it empty
% (as in \def\footerlink{}) to remove the link in the footer:
\def\footerlink{http://bradduthie.github.io/DuthieCV.pdf}

%\newcommand*\myglobe{%
% as suggested by Martin ;-)
%\includegraphics[height=1.6ex]{globe.pdf}}

%\fancypagestyle{specialfooter}{%
%  \fancyhf{}
%  \renewcommand\headrulewidth{0pt}
%  \fancyfoot[L]{Email: aduthie@abdn.ac.uk, brad.duthie@gmail.com 
%  }
%}

% The following metadata will show up in the PDF properties
\hypersetup{
  colorlinks = true,
  urlcolor = black,
  pdfauthor = {\name},
  pdfkeywords = {evolution, ecology, coexistence},
  pdftitle = {\name: Curriculum Vitae},
  pdfsubject = {Curriculum Vitae},
  pdfpagemode = UseNone
}

\geometry{
  body={6.5in, 8.5in},
  left=1.0in,
  top=1.25in
}

% Customize page headers
\pagestyle{myheadings}
\markright{\name}
\thispagestyle{empty}

% Custom section fonts
\usepackage{sectsty}
\sectionfont{\rmfamily\mdseries\Large}
\subsectionfont{\rmfamily\mdseries\itshape\large}

% Other possible font commands include:
% \ttfamily for teletype,
% \sffamily for sans serif,
% \bfseries for bold,
% \scshape for small caps,
% \normalsize, \large, \Large, \LARGE sizes.

% Don't indent paragraphs.
\setlength\parindent{0em}

% Make lists without bullets
\renewenvironment{itemize}{
  \begin{list}{}{
    \setlength{\leftmargin}{1.5em}
  }
}{
  \end{list}
}

\begin{document}

\thispagestyle{specialfooter}
% Place name at left


{\Huge \name}

\hrulefill

\vspace{0.03in}

I am an evolutionary ecologist and ecological modeller with broad interests in developing theory across the biological and environmental sciences. My research primarily uses mathematical and individual-based models to understand complex interactions in populations, communities, and social-ecological systems. I am especially interested in applying these modelling approaches to questions concerning the evolution and maintenance of biodiversity, and the management of populations under conservation conflict.


% Orcid ID (work in somewhere?):  0000-0001-8343-4995 
% Scopus Author ID:               55596393800


\hrulefill
% Alternatively, print name centered and bold:
%\centerline{\huge \bf \name}

\vspace{0.16in}

\begin{minipage}{0.65\linewidth}
\section*{Professional Appointments}
\begin{itemize}
	\item{\bf Lecturer in Environmental Modelling}, Nov. 2020 -- present \\
	Biological and Environmental Sciences, University of Stirling
\end{itemize}
\begin{itemize}
	\item{\bf Leverhulme Trust Early Career Fellow}, Nov. 2017 -- Oct. 2020 \\
	Biological and Environmental Sciences, University of Stirling
\end{itemize}
\begin{itemize}
	\item{\bf ERC Postdoctoral Research Assistant}, Nov. 2016 -- Oct. 2017 \\
	Biological and Environmental Sciences, University of Stirling
\end{itemize}
\begin{itemize}
	\item{\bf ERC Postdoctoral Research Fellow}, Jun. 2013 -- Oct. 2016 \\
	School of Biological Sciences, University of Aberdeen
\end{itemize}

\section*{Education}
\begin{itemize}
  \item {\bf Ph.D. Ecology and Evolutionary Biology}, Spring 2013 \\
	Graduate Minor: Statistics \\
	Certificate: Graduate Student Teaching Certificate \\
	Iowa State University
  \item {\bf B.S. Biology} (Magna cum laude), Spring 2007 \\
	Specialisation: Ecology, Evolution, Environment \\
	Southern Illinois University Edwardsville
  \item {\bf B.S. Philosophy} (Magna cum laude), Spring 2007 \\
	Minor: Chemistry \\
	Southern Illinois University Edwardsville
\end{itemize}


\end{minipage}
\begin{minipage}{0.35\linewidth}
\fcolorbox{black}{grey!10}{
 {\small
  \begin{tabular}{ll}
  \\ 
 {\bf
 A. Bradley Duthie} \\{\bf
 Ph.D., FHEA} \\
 Evolutionary Ecologist \\
 Ecological Modeller \\
 Biostatistician \\
 \\
 3A149 Cottrell Building \\
 University of Stirling \\
 Stirling \\
 FK9 4LA \\
 United Kingdom \\ \\ \\
 \setlength\tabcolsep{1.65pt}
 \begin{tabular}{ll}
 \Phone & (+44) 01786 467787 \\
 \Mobilefone & (+44) 07561 408101 \\
 \Envelope & \href{mailto:alexander.duthie@stir.ac.uk}{alexander.duthie@stir.ac.uk} \\
 & \href{mailto:brad.duthie@gmail.com}{brad.duthie@gmail.com} \\
 \Mundus~ & \href{http://bradduthie.github.io}{http://bradduthie.github.io}\\ 
 %\includegraphics[scale=0.08]{GitHub.pdf} & \href{https://github.com/bradduthie}{bradduthie}\\
 % \aiOrcidSquare & 0000-0001-8343-4995 
 \end{tabular}\\ \\
 \end{tabular}
 }
}
\end{minipage}



\section*{Book}
\begin{itemize}
\item[] {\bf Duthie, A. B.} 2025. \textit{Fundamental Statistical Concepts and Techniques in the Biological and Environmental \\ Sciences: With jamovi}. Chapman \& Hall/CRC, Boca Raton, USA. \href{https://bradduthie.github.io/stats/}{https://bradduthie.github.io/stats/}
\end{itemize}

\section*{Peer-reviewed journal articles}
\begin{etaremune}
\item {\bf Duthie, A. B.}, and V. J. Luque. 2025. Foundations of ecological and evolutionary change. \textit{Ecology and Evolution}. \textit{In press.}
\item Tomsin, I., {\bf A. B. Duthie}, N. Bunnefeld, H. Leirs, J. Casaer, and N. Beenaerts. 2025. Evaluating Management Scenarios for the European Hamster (\textit{Cricetus cricetus}) Using Quantitative Models. \textit{Ecology and Evolution}, 15:e72353. \href{https://doi.org/10.1002/ece3.72353}{DOI: 10.1002/ece3.72353}
\item Anderson, H. K., {\bf A. B. Duthie}, C. W. McDougall, R. S. Quilliam, and H. Price. 2025. The impact of the cost-of-living crisis on water poverty in Scotland: A lived-experience analysis. {\it Utilities Policy}. 96:101983. \href{https://doi.org/10.1016/j.jup.2025.101983}{DOI: 10.1016/j.jup.2025.101983}.
\item Fleming, M., A. Bell, H. R. Harison, J. Herrera, {\bf A. B. Duthie}, R. Kramer, and O. S. Rakotonarivo. 2025. Impact of price shocks and payments on crop diversification and forest use among Malagasy vanilla farmers. {\it Biological Conservation}. 302:110915. \href{https://www.sciencedirect.com/science/article/pii/S0006320724004774?dgcid=coauthor}{DOI: 10.1016/j.biocon.2024.110915}.
\item Bradfer-Lawrence, T., {\bf A. B. Duthie}, C. Abrahams, A. Mat\'{a}\v{s}, R. J. Barnett, A. Beeston, J. Darby, B. Dell, N. Gardner, A. Gasc, B. Heath, N. Howells, M. Janson, M.-V. Kyoseva, T. Luypaert, O. C. Metcalf, A. E. Nousek-McGregor, F. Poznansky, S. R. P.-J. Ross, S. Sethi, S. Smyth, E. Waddell, and J. S. P. Froidevaux. 2024. A practical manual for defining, generating and understanding current and future acoustic indices. {\it Methods in Ecology and Evolution}. \href{https://besjournals.onlinelibrary.wiley.com/doi/10.1111/2041-210X.14357}{DOI: 10.1111/2041-210X.14357}.
\item {\bf Duthie, A. B.}, R. Mangan, C. R. McKeon, M. C. Tinsley, and L. F. Bussi\`{e}re. 2023. resevol: An R package for spatially explicit models of pesticide resistance given evolving pest genomes. {\it PLoS Computational Biology}. 19 (12):e1011691. \href{https://journals.plos.org/ploscompbiol/article?id=10.1371/journal.pcbi.1011691}{DOI: 10.1371/journal.pcbi.1011691}
\item Fell, A., T. S. F. Silva, {\bf A. B. Duthie}, and D. Dent. 2023. A global systematic review of frugivorous animal tracking studies and the estimation of seed dispersal distances. {\it Ecology and Evolution}. 13:e10638. \href{https://onlinelibrary.wiley.com/doi/10.1002/ece3.10638}{10.1002/ece3.10638}
\item Bell, A. R., O. S. Rakotonarivo, A. Bhargava, {\bf A. B. Duthie}, W. Zhang, B. Sargent, S. Lewis, and A. Kipchumba. 2023. Payments don't reconcile agriculture and conservation. {\it Communications Earth and Environment}. 4:27. \href{https://www.nature.com/articles/s43247-023-00689-6}{10.1038/s43247-023-00689-6}
\item Marshall, B. M., and {\bf A. B. Duthie}. 2022. abmAnimalMovement: An R package for simulating animal movement using an agent-based model. {\it F1000 Research}. 11:1182. \newline \href{https://doi.org/10.5256/f1000research.137044.r164972}{DOI: 10.5256/f1000research.137044.r164972}.
\item Jones, I. L., A. Timoshenko, I. Zuban, K. Zhadan, J. J. Cusack, {\bf A. B. Duthie}, I. D. Hodgson, J. Minderman, R. A. Pozo, R. C. Whytock, and N. Bunnefeld. 2022. Achieving international biodiversity targets: learning from migratory waterfowl management. {\it Journal of Applied Ecology}. 59:1911-1924. \href{https://besjournals.onlinelibrary.wiley.com/doi/10.1111/1365-2664.14198}{DOI: 10.1111/1365-2664.14198}.
\item Bach, A., J. Minderman, N. Bunnefeld, A. Mill, and {\bf A. B. Duthie}. 2022. Intervene or wait? Modelling the timing of intervention in conservation conflicts adaptive management under uncertainty. {\it Ecology and Society}. 27 (3):3. \href{https://ecologyandsociety.org/vol27/iss3/art3/}{DOI: 10.5751/ES-13341-270303}.
\item Dawson, S., C. P\'{e}rez Carmona, M. Gonzalez Suarez, M. J\"{o}nsson, F. de Carvalho, M. Mallen-Cooper, Y. Melero, H. Moor, J. Simaika, and {\bf A. B Duthie}. 2021. The traits of `trait ecologists': an analysis of the use of trait and functional trait terminology. {\it Ecology and Evolution}. 11:16434-16445. \href{https://onlinelibrary.wiley.com/doi/10.1002/ece3.8321}{DOI: 10.1002/ece3.8321}.
\item Rakotonarivo, O. S., A. Bell, B. Dillon, {\bf A. B. Duthie}, A. Kipchumba, R. Rasolofoson, J. Razafimanahaka, and N. Bunnefeld. 2021. Experimental evidence on the impact of payments and property rights on forest user decisions. {\it Frontiers in Conservation Science}. 2:661987. \href{https://doi.org/10.3389/fcosc.2021.661987}{DOI: 10.3389/fcosc.2021.661987}.
\item {\bf Duthie, A. B.}, J. Minderman, O. S. Rakotonarivo, G. Ochoa, and N. Bunnefeld. 2021. Online multiplayer games as virtual laboratories for collecting data on socio-ecological decision-making. {\it Conservation Biology}. 35:1051-1053. \href{https://doi.org/10.1111/cobi.13633}{DOI: 10.1111/cobi.13633}
\item Rakotonarivo, O. S., A. Bell, K. Abernathy, J. Minderman, {\bf A. B. Duthie}, S. Redpath, A. Keane, H. Travers, S. Bourgeois, L.-L. Moukagni, J. J. Cusack, I. L. Jones, R. A. Pozo, and N. Bunnefeld. 2021. The role of incentive-based instruments and social equity in conservation conflict interventions. {\it Ecology and Society}. 26:8. \href{https://www.ecologyandsociety.org/vol26/iss2/art8/}{DOI: 10.5751/ES-12306-260208}
\item Pozo, R. A., E. LeFlore, {\bf A. B. Duthie}, N. Bunnefeld, I. L. Jones, J. Minderman, O. S. Rakotonarivo, and J. J. Cusack. 2021. A spatiotemporal, multi-species assessment o10.1111/1365-2664.14198f wildlife impacts on local community livelihoods. {\it Conservation Biology}. 35:297-306. \href{https://doi.org/10.1111/cobi.13565} {DOI: 10.1111/cobi.13565}
\item Rakotonarivo, O. S., I. L. Jones, A. Bell, {\bf A. B. Duthie}, J. J. Cusack, J. Minderman, J. Hogan, I. Hodgson, and N. Bunnefeld. 2021. Experimental evidence for conservation conflict interventions: the importance of financial payments, community trust and equity attitudes. {\it People and Nature}. 3: 162-175. \href{https://doi.org/10.1002/pan3.10155}{DOI: 10.1002/pan3.10155}
\item Nilsson, L., J. Minderman, N. Bunnefeld, and {\bf A. B. Duthie}. 2021. Effects of stakeholder empowerment on crane population and agricultural production. {\it Ecological Modelling}. 404:109396. \href{https://doi.org/10.1016/j.ecolmodel.2020.109396}{DOI: 10.1016/j.ecolmodel.2020.109396}
\item {\bf Duthie, A. B.} 2020. Component response rate variation underlies the stability of highly complex finite systems. {\it Scientific Reports}. 10: 8296. \href{https://www.nature.com/articles/s41598-020-64401-w}{DOI: 10.1038/s41598-020-64401-w}
\item Cusack, J. J., {\bf A. B. Duthie}, I. L. Jones, J. Minderman, R. A. Pozo, O. S. Rakotonarivo, S. Redpath, and N. Bunnefeld. 2020. Integrating conflict, lobbying and compliance to predict the sustainability of natural resource use. {\it Ecology and Society}. 25(2): 13. \href{https://doi.org/10.5751/ES-11552-250213}{DOI: 10.5751/ES-11552-250213}
\item Minderman, J., J. J. Cusack, {\bf A. B. Duthie}, I. L. Jones, R. A. Pozo, O. S. Rakotonarivo, and N. Bunnefeld. 2019. Decision trees for data publishing may exacerbate conservation conflict. {\it Nature Ecology and Evolution}. 3:318. \href{https://doi.org/10.1038/s41559-019-0804-7}{DOI: 10.1038/s41559-019-0804-7}
\item Cusack, J. J., {\bf A. B. Duthie}, O. S. Rakotonarivo, R. A. Pozo, T. H. E. Mason, J. M\r{a}nsson, L. Nilsson, R. McKenzie, I. M. Tombre, E. Eyth\'{o}rsson, J. Madsen, A. Tulloch, G. Churchill, J. Shaw, R. D. Hearn, S. Redpath, and N. Bunnefeld. 2019. Time series analysis reveals synchrony and asynchrony between conflict management effort and increasing large grazing bird populations in northern Europe. {\it Conservation Letters}. 12:e12450. \href{http://onlinelibrary.wiley.com/doi/10.1111/conl.12450/full}{DOI: 10.1111/conl.12450}
\item {\bf Duthie, A. B.}, J. J. Cusack, J. Minderman, I. L. Jones, E. B. Nilsen, R. A. Pozo, O. S. Rakotonarivo, B. Van Moorter, and N. Bunnefeld. 2018. GMSE: an R package for generalised management strategy evaluation. {\it Methods in Ecology and Evolution}. 9:2396-2401. \href{https://besjournals.onlinelibrary.wiley.com/doi/10.1111/2041-210X.13091}{DOI: 10.1111/2041-210X.13091} 
\item Paine, C. E. T., A. Deasey, and {\bf A. B. Duthie}. 2018. Towards general mechanistic predictions of community dynamics. {\it Functional Ecology}. 32 (7):1681-1692. \href{http://onlinelibrary.wiley.com/doi/10.1111/1365-2435.13096/pdf}{DOI: 10.1111/1365-2435.13096}
\item Redpath, S. M., Andren, Z. Baynham-Herd, N. Bunnefeld, {\bf A. B. Duthie}, Frank, C. A. Garcia, A. Keane, J. M\r{a}nsson, L. Nilsson, C. R. J. Pollard, O. S. Rakotonarivo, C. F. Salk, and H. Travers. 2018. Games as tools to address conservation conflicts. {\it Trends in Ecology and Evolution}. 33 (6):415-426. \href{https://www.cell.com/trends/ecology-evolution/fulltext/S0169-5347(18)30059-4}{DOI: 10.1016/j.tree.2018.03.005}
\item {\bf Duthie, A. B.}, G. Bocedi, R. R. Germain, and J. M. Reid. 2018. Evolution of pre-copulatory and post-copulatory strategies of inbreeding avoidance and associated polyandry. {\it Journal of Evolutionary Biology}. 31:31-45. \href{https://onlinelibrary.wiley.com/doi/10.1111/jeb.13189/full}{DOI: 10.1111/jeb.13189}
\item {\bf Duthie, A. B.}, A. M. Lee, and J. M. Reid. 2016. Inbreeding parents should invest more resources in fewer offspring. {\it Proceedings of the Royal Society B}. 282:20161845. \href{http://rspb.royalsocietypublishing.org/content/283/1843/20161845}{DOI: 10.1098/rspb.2016.1845}
\item {\bf Duthie, A. B.}, and J. M. Reid. 2016. Evolution of inbreeding avoidance and inbreeding preference through mate choice among interacting relatives. {\it American Naturalist}. 188 (6):651-667. \href{http://www.journals.uchicago.edu/doi/full/10.1086/688919}{DOI: 10.1086/688919}
\item {\bf Duthie, A. B.}, and J. D. Nason. 2016. Plant connectivity underlies plant-pollinator-exploiter distributions in {\it Ficus petiolaris} and associated pollinating and non-pollinating fig wasps. {\it Oikos}. 125 (11):1597-1606. \href{http://onlinelibrary.wiley.com/doi/10.1111/oik.02905/abstract}{DOI: 10.1111/oik.02905}
\item {\bf Duthie, A. B.}, G. Bocedi, and J. M. Reid. 2016. When does female multiple mating evolve to adjust inbreeding? Effects of inbreeding depression, direct costs, mating constraints, and polyandry as a threshold trait. {\it Evolution}. 70 (9):1927-1943. \href{http://onlinelibrary.wiley.com/doi/10.1111/evo.13005/abstract}{DOI: 10.1111/evo.13005}
\item Frater, P. N., and {\bf A. B. Duthie}. 2016. Power scaling, vascular branching patterns, and the golden ratio. {\it Ideas in Ecology and Evolution}. 9:15-18. DOI: \href{http://ojs.library.queensu.ca/index.php/IEE/article/view/6312}{10.4033/iee.2016.9.4.n}
\item Reid, J. M., G. Bocedi, P. Nietlisbach, {\bf A. B. Duthie}, M. E. Wolak, E. A. Gow, and P. Arcese. 2016. Variation in parent-offspring kinship in socially monogamous systems with extra-pair reproduction and inbreeding. {\it Evolution}. 70 (7):1512-1529. \href{http://onlinelibrary.wiley.com/doi/10.1111/evo.12953/abstract}{DOI: 10.1111/evo.12953}
\item Reid, J. M., P. Arcese, G. Bocedi, {\bf A. B. Duthie}, M. E. Wolak, and L. F. Keller. 2015. Resolving the conundrum of inbreeding depression but no inbreeding avoidance: estimating sex-specific selection on inbreeding by song sparrows ({\it Melospiza melodia}). {\it Evolution}. 69 (11):2846-2861. \href{http://onlinelibrary.wiley.com/doi/10.1111/evo.12780/abstract}{DOI: 10.1111/evo.12780}
\item {\bf Duthie, A. B.}, K. C. Abbott, and J. D. Nason. 2015. Trade-offs and coexistence in fluctuating environments: evidence for a key dispersal-fecundity trade-off in five nonpollinating fig wasps. {\it American Naturalist}. 186 (1):151-158. \href{http://www.jstor.org/stable/10.1086/681621}{DOI: 10.1086/681621}
\item Reid, J. M., {\bf A. B. Duthie}, M. E. Wolak, and P. Arcese. 2015. Demographic mechanisms of inbreeding adjustment through extra-pair reproduction. {\it Journal of Animal Ecology}. 84 (4):1029-1040. \href{http://onlinelibrary.wiley.com/doi/10.1111/1365-2656.12340/abstract}{DOI: 10.1111/1365-2656.12340}
\item {\bf Duthie, A. B.}, and J. M. Reid. 2015. What happens after inbreeding avoidance? Inbreeding by rejected relatives and the inclusive fitness benefit of inbreeding avoidance. {\it PLoS One}. 10 (4):e0125140. \href{http://journals.plos.org/plosone/article?id=10.1371/journal.pone.0125140}{DOI: 10.1371/journal.pone.0125140}
\item Duthie, A. C., and {\bf A. B. Duthie}. 2015. Do music and art influence one another? Measuring cross-modal similarities in music and art. {\it Polymath: An Interdisciplinary Arts and Sciences Journal}. \href{https://ojcs.siue.edu/ojs/index.php/polymath/article/view/3013}{5 (1):1-22}.
\item Reid, J. M., P. Arcese, L. F. Keller, R. R. Germain, {\bf A. B. Duthie}, S. Losdat, M. E. Wolak, and P. Nietlisbach. 2015. Quantifying inbreeding avoidance through extra-pair reproduction. {\it Evolution}. 69 (1):59-74. \href{http://onlinelibrary.wiley.com/doi/10.1111/evo.12557/abstract}{DOI: 10.1111/evo.12557}
\item {\bf Duthie, A. B.}, K. C. Abbott, and J. D. Nason. 2014. Trade-offs and coexistence: A lottery model applied to fig wasp communities. {\it American Naturalist}. 183 (6):826-841. \href{http://www.jstor.org/stable/10.1086/675897}{DOI: 10.1086/675897}
\item {\bf Duthie, A. B.}, and M. R. Falcy. 2013. The influence of habitat autocorrelation on plants and their seed-eating pollinators. {\it Ecological Modelling}. 251:260-270. \href{http://www.sciencedirect.com/science/article/pii/S0304380013000021}{DOI: 10.1016/j.ecolmodel.2012.12.019}
\item {\bf Duthie, A. B.} 2004. The fork and the paperclip: A memetic perspective. {\it Journal of Memetics - Evolutionary Models of Information Transmission}. 8 (1). \href{http://cfpm.org/jom-emit/2004/vol8/duthie_ab.html}{http://cfpm.org/jom-emit/2004/vol8/duthie\_ab.html}.
\end{etaremune}

% Report section somehow:
% Bunnefeld, N., Pozo, R.A., Cusack, J.J., Duthie, A.B. & Minderman, J. 2020. Development of a  population  model  tool  to  predict  shooting  levels  of  Greenland  barnacle  geese  on  Islay.  Scottish Natural Heritage Research Report No. 1039.

\section*{Grants and Fellowships}
\vspace{-5mm}
\begin{longtable}{p{0.85\linewidth} | p{0.15\linewidth}}
\textbf{Primary investigator} & \\

UKRI cross research council responsive mode pilot scheme: ``Ecological Knowledge &   1,198,180 GBP \\
$\hspace{4mm}$Games'' (2024) & \\
Joint Call CESAB- sDiv SYNERGY: Coexistence and stability in high-diversity &  30,000 EUR \\
$\hspace{4mm}$communities. ``Unification of modern Coexistence theory and Price equation'' (2020) &  \\
Leverhulme Trust Early Career Fellowship. ``Biodiversity and ecosystem function in & 88,650 GBP \\
$\hspace{4mm}$ephemeral patches: a trait-based approach'' (2017) &  \\
Biotechnology Graduate Fellowship (ISU; 2007) &  20,000 USD \\
& \\
\textbf{Co-investigator} & \\
European Commission (Horizon Europe). ``Restoring Ecosystems to Stop the Threat of & 332,245 GBP \\
$\hspace{4mm}$(Re-) Emerging Infectious Diseases (RestoreID)'' (2023) &  \\
Atlantic Salmon Trust. ``Decision Support Tool'' (2021) & 12,000 GBP \\
Joint Newton funded international partnership between the Biotechnology and Biological & 620,956 GBP \\
$\hspace{4mm}$Sciences Research Council (BBSRC) in the UK and the S\~{a}o Paulo Research & \\
$\hspace{4mm}$Foundation (FAPESP). ``Enhancing Diversity to Overcome Resistance Evolution'' (2019) & \\
Scottish Natural Heritage Tender. ``Development of a population model tool to predict & 9,947 GBP \\ 
$\hspace{4mm}$shooting levels of Greenland barnacle geese on Islay'' (2018) &  \\
National Science Foundation (USA) Doctoral Dissertation Improvement Grant. & 13,411 USD \\ 
$\hspace{4mm}$``Interspecific interactions over a latitudinal resource gradient'' (2010) &  \\
\end{longtable}

%\item BBSRC International partnership (Co-PI; 2022) \hfill 34,137 GBP
%\item BBSRC International partnership (Co-PI; 2022) \hfill 25,571 GBP


\section*{Teaching Experience}
\hrulefill

%\vspace{0.06in}

As an educator, I take a learning-centred approach to guide students through course material, and I evaluate my teaching as successful when I have evidence that students are thinking actively and effectively about biological concepts. I have experience developing and delivering course content, and in student assessment, in both traditional and online formats. I have additionally developed \href{https://bradduthie.shinyapps.io/EcoEdu/}{learning software}, specifically for teaching numerical and individual-based ecological modelling to undergraduate students.

\hrulefill

%\vspace{0.03in}

\begin{itemize}
\item {\bf Professional Development:}
\begin{itemize}
\item[$\bullet$]{{\bf Fellow of the Higher Education Academy}}
\item[$\bullet$]{Principles of Learning and Teaching in Higher Education programme. University of Aberdeen.}
\item[$\bullet$]{Preparing Future Faculty (PFF) Program: Iowa State University. PFF Associate.}
\item[$\bullet$]{Graduate Student Teaching Certificate (GSTC) Program (PFF Track): Iowa State University.}
\end{itemize}

\item {\bf Programme Director:}
\begin{itemize}
\item[$\bullet$]{{\it Biology}. University of Stirling. 2023-present.}
\item[$\bullet$]{{\it Animal Biology}. University of Stirling. 2023-present.}
\item[$\bullet$]{{\it Biology and Mathematics}. University of Stirling. 2023-present.}
\end{itemize}

\item {\bf University Instruction:}
\begin{itemize}
\item[$\bullet$]{{\it Statistical Techniques -- Module Coordinator} (SCIU4T4; Undergraduate: 20 credits) University of Stirling. Spring 2021-2025.}
\item[$\bullet$]{{\it Evolution and Genetics -- Module Coordinator} (BIOU3GE; Undergraduate: 20 credits) University of Stirling. Autumn 2021-2025.}
\item[$\bullet$]{{\it Population and Community Ecology -- Module Contributor} (BIOU9PC; Undergraduate: 20 credits) University of Stirling. Autumn 2024-2025.}
\item[$\bullet$]{{\it Fundamentals for Scientific Inquiry -- Module Contributor} (BIOU9PC; Undergraduate: 20 credits) University of Stirling. Autumn 2025.}
\item[$\bullet$]{{\it Biology Field Course -- Module Contributor} (SCIU3FB; Undergraduate: 20 credits) University of Stirling. Autumn 2017-2025.}
\item[$\bullet$]{{\it Behavioural Ecology -- Module Contributor} (BIOU6BE; Undergraduate: 20 credits) University of Stirling. Spring 2019-2024.}
\item[$\bullet$]{{\it Introduction to Ecological and Environmental Modelling -- Module Contributor} (BI 4803; Undergraduate: 15 credits). University of Aberdeen. Spring 2015, 2016.}
\item[$\bullet$]{{\it Biology for Undergraduates (BUGS) -- Tutorial Session ZOO-4} (BI 1006; Undergraduate: 7.5 credits). University of Aberdeen. Autumn 2014.}
\item[$\bullet$]{{\it Biological Evolution} (BIOL 315; Undergraduate: 3 credits). Iowa State University. Summer 2011, 2012*, Spring 2013*. *Distance Learning Course (online).}
\end{itemize}

\item {\bf University Teaching Assistantships:}
\begin{itemize}
\item[$\bullet$]{{\it Biological Evolution} (BIOL 315; Undergraduate: 3 credits). Iowa State University. Spring 2011-2012.}
\item[$\bullet$]{{\it Principles of Biology Laboratory I} (BIOL 211L; Undergraduate: 1 credit). Iowa State University. Autumn 2008, 2012.}
\item[$\bullet$]{{\it Principles of Biology Laboratory II} (BIOL 212L; Undergraduate: 1 credit). Iowa State University. Autumn 2009-2011, Spring 2008-2010.}
\end{itemize}

\item {\bf Workshops:}
\begin{itemize}
\item[$\bullet$]{{\it \href{https://stirlingcodingclub.github.io/studyGroup}{Stirling Coding Club (SCC).}} Developing and leading fortnightly workshops (13 total) on programming and data analysis at the University of Stirling (annually delivered 2018-present).}
\item[$\bullet$]{{\it \href{http://bradduthie.github.io/talks/BCB_talk.pdf}{Evolution of biopesticide resistance with agent-based modelling in the resevol package}}. Biology, Computing Science, and Business (BCB) Society Event. University of Stirling. 16 NOV 2022.}
\item[$\bullet$]{{\it \href{http://bradduthie.github.io/notes/R_intro_notes.html}{Introduction to R programming}} (Workshop leader for first year PhD students). IAPETUS Doctoral Training Partnership Event. Scottish Centre for Ecology and the Natural Environment. 12 MAR 2024.}
\item[$\bullet$]{{\it \href{https://bradduthie.github.io/version_control/vc_notes.html}{Version control for reproducible science}} (Workshop leader for later year PhD students). IAPETUS Doctoral Training Partnership Event. Scottish Centre for Ecology and the Natural Environment. 16 JAN 2020.}
\item[$\bullet$]{{\it \href{https://bradduthie.github.io/blog/Manuscripts-in-Rmarkdown/}{Exploring the power of `R Shiny' and `R Markdown'}} (Workshop leader for later year PhD students). IAPETUS Doctoral Training Partnership Induction Event. University of Stirling. 22 NOV 2018.}
\end{itemize}

\end{itemize}

\section*{Supervision}
\begin{itemize}
\item {\bf Undergraduate researchers}. 34 Students (2010-2025) %Cody Duncan (2010), Megan Stoner (2010-2011), Elizabeth Howe (2011), Christina Pacholec (2011), Ethan Van Helten (2011), Christine Lim (2011-2012), Aisha Azhar (2012-2013), Arianna Chiti (2018-2020), Rose McKeon (The Genetics Society Summer Studentship, 1600 GBP; 2019-2020), Erin Iffla (2020-2021), Daniel Leask (2020-2021), Megan Reid (2020-2021), Hillary Yuen (2021-2022), Sara De Diego Herrera (2021-2022), Sophie Barker (2021-2022), Molly Davies (2022-2023), Callum Laing (2022-2023), Elisavet Mitromeleti (2022-2023), and Brandon Bisono (2022-2023), Christina Zaglis (2023-2024), Alex Hooker Niembro (2023-2024), Chris McPherson (2023-2024), Kai Galas (2023-2024), Matt Govans (2024-2025), Zoe Kwiatkowski (2024-2025), William Dlugolecki (2024-2025), Christian Monaco (2024-2025), Izak Maas (2024-2025).
%\item {\bf Masters researchers}. Oluyomi Olusomidomo (2022), and Doris Chinonyerem Opara (2022), Ishrat Fatima (2023).
\item {\bf Masters researchers}. 6 Students (2022-2024)
\item {\bf Doctoral researchers} (primary). A Bach (2018-2023), TSC Pannetier (2020-2022), JH Paterson (2020-2022), A Fell (2022-2023), and BM Marshall (2021-2025), C Corcoran (2024-2028), M Beltran (2025-2029).
\item {\bf Doctoral researchers} (secondary). R Aitchison (2023-2027), A Costley-Wood (2023-2027), A Chiti (2023-2027), M Simmonds (2023-2027), S Paplauskas (2023-2025), AP Charmouh (2019-2023).
\item {\bf Postdoctoral researchers}. A Fell (2024-2025), M Samuel (2025-2026), T Cimpeanu (2025-2026).
\end{itemize}

\section*{Leadership}
\begin{itemize}
\item {\bf Associate Editor}: Journal of Animal Ecology (2021-present; 27 manuscripts handled)
\item {\bf Manuscript reviewer}: {\it Acta Oecologica}, {\it American Naturalist}, {\it Annals of Human Biology}, {\it Arthropod-Plant Interactions}, {\it Behavioral Ecology} (2), {\it Behavioral Ecology and Sociobiology} (4), {\it Biology Letters} (5), {\it BMC Evolutionary Biology}, {\it Communications Biology}, {\it Ecological Complexity}, {\it Ecological Modelling} (4), {\it Ecology and Evolution} (4), {\it Ecology and Society}, {\it Environmental Conservation}, {\it Evolution} (6), {\it Evolutionary Ecology} (3), {\it Evolution Letters} (2), {\it Insects} (2), the {\it Journal of Animal Ecology}, the {\it Journal of Applied Ecology} (3), the {\it Journal of Theoretical Biology} (3), {\it Landscape Ecology}, {\it Molecular Ecology} (2), {\it Oikos} (5), {\it PLoS One}, {\it Proceedings of the National Academy of Sciences USA} (2), {\it Proceedings of the Royal Society B} (2), {\it Royal Society Open Science}, {\it Scientific Reports} (6), and {\it Symbiosis}
\item {\bf Grant reviewer}: National Science Foundation (USA), Universit\'{e} Bourgogne -- Franche-Comt\'{e} (UBFC), Centre for Interdisciplinary Research in Animal Health (CIISA), and L'Agence nationale de la recherche (ANR)
\item {\bf Conference organiser}: Midwest Ecology and Evolution Conference at Iowa State University (2010)
\item {\bf Workshop organiser}: European Congress of Conservation Biology in Jyv\"{a}skyl\"{a}, Finland (2018)
\item {\bf Doctoral examiner}: Internal (2), External (1), Independent Chair (2)
\item {\bf Representative}: Biological and Environmental Sciences Library (UoS; 2019-present) and AdvanceHE (UoS; 2021-present), Post-doctoral researchers (organiser at UoS; 2017-2019), Graduate and Professional Student Senate (Senator at ISU; 2011-2013), and Graduate Research in Evolutionary Biology and Ecology (Vice President at ISU; 2009-2010)
\end{itemize}

\section*{Computer Skills}
\begin{itemize}
\item Experience in statistical analysis, mathematical \& individual-based modelling, genetic algorithms, and computer programming in C and R (including Rshiny, Bookdown, \& Rmarkdown) %, and MATLAB
\item Use of Linux, git, and high performance computing clusters
\item Knowledge of HTML, Markdown, \LaTeX, and MS Office Suite
\item Course design and instruction in Blackboard and Canvas
%\item Course instruction in WebCT
\end{itemize}

\section*{Software Developed}

All listed software is publicly available on the \href{https://cran.r-project.org/}{Comprehensive R Archive Network} (CRAN) and on \href{https://github.com/bradduthie}{GitHub}.

\begin{etaremune}
\item {\bf resevol} {\it v0.2.0.0}. 2021. Simulate Agricultural Production and Evolution of Pesticide. \\
{\it Written in R and C.} \\
R Package link:  \href{http://CRAN.R-project.org/package=resevol}{< http://CRAN.R-project.org/package=resevol >} \\
Website link: \href{https://bradduthie.github.io/resevol/}{< https://bradduthie.github.io/resevol/ >}
\item {\bf GMSE} {\it v0.3.1.9}. 2017. Generalised Management Strategy Evaluation Simulator. {\it Written in R and C.} \\
R package link: \href{http://CRAN.R-project.org/package=GMSE}{< http://CRAN.R-project.org/package=GMSE >} \\
Website link: \href{https://confoobio.github.io/gmse/}{< https://confoobio.github.io/gmse/ >}
\item {\bf gamesGA} {\it v1.1.3.2}. 2017. Genetic Algorithm for Sequential Symmetric Games. {\it Written in R and C.} \\
R package link: \href{http://CRAN.R-project.org/package=gamesGA}{< http://CRAN.R-project.org/package=gamesGA >}
\end{etaremune}

\section*{Selected Awards}
\begin{itemize}
\item $\ddagger$Denotes awarded annually to one UoS staff member
\item $\ddagger$ University of Stirling Recognising and Advancing Teaching Excellence (RATE): Outstanding Commitment to Equality, Diversity, and Inclusion Award winner (2024, 2025)
\item $\ddagger$ University of Stirling Recognising and Advancing Teaching Excellence (RATE): Excellence in Teaching in the Faculty of Natural Sciences Award winner (2023), honourable mention (2022);
\item $\ddagger$ University of Stirling Recognising and Advancing Teaching Excellence (RATE): Fantastic Feedback Award winner (2022)
\item Outstanding Research Activity Dedicated to Equality, Diversity and Inclusion (highly commended as part of the VIBE team; 2024)
\item Outstanding Activity Dedicated to Enhancing Research Culture (highly commended; 2023)
\item Iowa State University Teaching Excellence Award (top 10\% of TAs; 2010)
\item Donal G. Myer Outstanding Student Award (SIUE; awarded to one science student; 2007)
\item Ober Honours Award in Philosophy (SIUE; awarded to one philosophy student; 2007)
\item Phi Kappa Phi (Collegiate Honour Society; 2006)
\item Chancellor's Scholarship (SIUE; full tuition for undergraduate education; 2003-2007)
\end{itemize}

\section*{Presentations}
\begin{itemize}
\item $\ddagger$Denotes invited talk, *Denotes undergraduate author
\item {\bf Duthie, A. B.} and V. J. Luque. 2025. Darwin's Dream: Unifying Ecological and Evolutionary Change. {\bf International Society for the History, Philosophy, and Social Studies of Biology}, Porto, Portugal.
\item $\ddagger${\bf Duthie, A. B.} and V. J. Luque. 2024. Foundations of ecological and evolutionary change. Institut Universitari Cavanilles de Biodiversitat i Biologia Evolutiva, University of Val\'{e}ncia, Val\'{e}ncia, Spain.
\item {\bf Duthie, A. B.}, R. Mangan, C. R. McKeon, M. C. Tinsley, and L. F. Bussi\'{e}re. 2022. \href{http://bradduthie.github.io/talks/BES2022.pdf}{resevol: an R package for spatially explicit models of pesticide resistance given evolving pest genomes}. {\bf British Ecological Society Meeting}, Edinburgh, Scotland.
\item Cusack, J., {\bf A. B. Duthie}, R. Pozo, S. Redpath, and N. Bunnefeld. 2019. \href{http://bradduthie.github.io/talks/duthie_ISEM_2019.pdf}{Non-compliance and biased management decisions predict population trends of harvested species}. {\bf International Society for Ecological Modelling 2019}, Salzburg, Austria.
\item Dawson, S., {\bf A. B. Duthie}, M. Gonzalez-Suarez, M. J\"{o}nsson, C. P\'{e}rez Carmona, F. Chichorro de Carvalho, M. Mallen-Cooper, Y. Melero, H. Moor, and J. Simaika. 2019. A survey on the interpretation and application of the terms `trait' and `functional trait' among ecologists. {\bf Ecological Society of America 2019}, Louisville, Kentucky, USA.
\item $\ddagger${\bf Duthie, A. B.}, J. J. Cusack, O. S. Rakotonarivo, and N. Bunnefeld. 2019. Generalised Management Strategy Evaluation: modelling biodiversity and food security. {\bf Centre for Biodiversity Dynamics Seminar} at the Norwegian University of Science and Technology, Trondheim, Norway.
\item {\bf Duthie, A. B.} 2018. \href{https://www.dropbox.com/s/v42iqdccpkyg55w/Duthie_ECCB2018.pdf?dl=0}{Ecological inference from functional traits}. {\bf Fifth European Congress of Conservation Biology} in Jyv\"{a}skyl\"{a}, Finland.
\item $\ddagger${\bf Duthie, A. B.} 2018. When, why, and how to avoid, prefer, or tolerate inbreeding. {\bf Institute of Evolutionary Biology Seminar} at the University of Edinburgh, Edinburgh, Scotland.
\item $\ddagger${\bf Duthie, A. B.} 2017. GMSE: a general tool for management strategy evaluation. {\bf Mathematics and Statistics Group Seminar} at the University of Stirling, Stirling, Scotland.
\item $\ddagger${\bf Duthie, A. B.} 2017. \href{https://bradduthie.shinyapps.io/GRIMSO}{Introduction to genetic algorithms, and their potential to link complex games to real-world behaviour in conservation conflicts}. {\bf Workshop: ``Behavioural games and conflict''} at Grims\"{o} Wildlife Research Station, Grims\"{o}, Sweden.
\item {\bf Duthie, A. B.}, G. Bocedi, and J. M. Reid. 2015. Evolution of polyandry and inbreeding avoidance: a genetically-explicit finite population model. {\bf Population Genetics Group Meeting 49} at the University of Edinburgh, Edinburgh, Scotland.
\item {\bf Duthie, A. B.}, G. Bocedi, and J. M. Reid. 2015. Should females mate multiply to avoid inbreeding? Insights from a computational modelling approach. {\bf The Scottish Informatics and Computer Science Alliance (SICSA) workshop on Computational Ecology} at the University of Edinburgh, Edinburgh, Scotland.
\item {\bf Duthie, A. B.} and J. M. Reid. 2014. Evolution of inbreeding and inbreeding avoidance among multiple relatives. {\bf Workshop: ``Evolution of Mating Systems''} at the Konnevesi Research Station of the University of Jyv\"{a}skyl\"{a}, Konnevesi, Finland.
\item {\bf Duthie, A. B.}, Lim*, C., and J. D. Nason. 2013. The effects of life history characteristics on male dimorphism frequency in two species of non-pollinating fig wasps. {\bf Ento '13 International Symposium and Annual National Science Meeting} at the University of Saint Andrews, Saint Andrews, Scotland.
\item {\bf Duthie, A. B.}, K. C. Abbott, and J. D. Nason. 2013. \href{https://www.youtube.com/watch?v=8Oi48FdaLXY}{A fluctuating environment drives coexistence in five non-pollinating fig wasps}. {\bf XIV Conference of the European Society for Evolutionary Biology} at the University of Lisbon, Lisbon, Portugal.
\item Lim*, C., {\bf A. B. Duthie}, and J. D. Nason. 2012. Violence inside the fig: Sexual selection among two species of non-pollinating male fig wasps. {\bf Symposium on Undergraduate Research and Creative Expression} at Iowa State University, Ames, Iowa.
\item {\bf Duthie, A. B.}, K. C. Abbott, and J. D. Nason. 2012. Does the storage effect maintain coexistence in non-pollinating fig wasp communities? {\bf Nineteenth Annual GREBE Spring Symposium} at Iowa State University, Ames, Iowa.
\item {\bf Duthie, A. B.} 2010. Turn-taking in habitat patch selection can be evolutionarily stable. {\bf Midwest Ecology and Evolution Conference} at Iowa State University, Ames, Iowa.
\item Day, K., {\bf A. B. Duthie}, and J. D. Nason. 2009. Cryptic parasites and their effects on the fitness and evolutionary stability of the fig-pollinator mutualism. {\bf Evolution} at the University of Idaho, Moscow, Idaho.
\item {\bf Duthie, A. B.} and M. Falcy. 2009. Spatial heterogeneity promotes coexistence in a mutualist/exploiter community. {\bf Sixteenth Annual GREBE Spring Symposium} at Iowa State University, Ames, Iowa.
\item {\bf Duthie, A. B.} and J. D. Nason. 2008. Dynamics of a Sonoran fig wasp community: A graphical modeling approach. {\bf Evolution} at the University of Minnesota, Minneapolis, Minnesota.
\item {\bf Duthie, A. B.} and J. D. Nason. 2008. Graphical modeling and the community dynamics of a sonoran desert fig, its pollinator, and exploiter wasps. {\bf Fifteenth Annual GREBE Spring Symposium} at Iowa State University, Ames, Iowa.
\item {\bf Duthie, A. B.} and M. Smith. 2007. Environmental effects on achenes of {\it Boltonia decurrens}, a threatened floodplain species: Age, mass, germination, and viability. {\bf Illinois State Academy of Sciences Annual Meeting} at the Illinois State Museum, Springfield, Illinois.
\item {\bf Duthie, A. B.}. 2006. This brain owns itself: How things like us project a unified agent {\bf First Annual Undergraduate Philosophy Conference} at Southern Illinois University Edwardsville, Edwardsville, Illinois.
\end{itemize}

\section*{Posters}
\begin{itemize}
\item {\bf Duthie, A. B.} 2024. \href{http://bradduthie.github.io/posters/duthie_UKCOTS_2024.pdf}{Fundamental Statistical Concepts and Techniques in the Biological and Environmental Sciences: With jamovi}. {\bf UK Conference on Teaching Statistics (UKCOTS)}, Manchester, England.
\item {\bf Duthie, A. B.} 2021. \href{http://bradduthie.github.io/posters/duthie_BES_2021.pdf}{Videogames as virtual laboratories for collecting decision-making data to understand challenging social-ecological problems}. {\bf Ecology Across Borders:} Joint British Ecological Society and French Society for Ecology and Evolution, Liverpool, England (virtual attendance and eposter).
\item {\bf Duthie, A. B.} 2019. \href{http://bradduthie.github.io/posters/duthie_ISEM_2019.pdf}{Component response rate variation underlies the stability of complex systems}. {\bf International Society for Ecological Modelling 2019}, Salzburg, Austria.
\item {\bf Duthie, A. B.}, and J. M. Reid. 2015. The effect of relatedness structure and sexual conflict on the evolution of inbreeding avoidance and preference {\bf XV Conference of the European Society for Evolutionary Biology} in Lausanne, Switzerland.
\item {\bf Duthie, A. B.}, K. C. Abbott, and J. D. Nason. 2012. Does the storage effect maintain coexistence in non-pollinating fig wasp communities? {\bf Evolution} in Ottawa, Canada.
\item {\bf Duthie, A. B.}, K. C. Abbott, and J. D. Nason. 2011. Coexistence in fig wasp communities: A lottery model. {\bf Workshop 5: Coevolution and the ecological structure of plant-insect communities} at the Mathematical Biosciences Institute in Columbus, Ohio.
\item {\bf Duthie, A. B.}, and M. R. Falcy. 2011. Spatial heterogeneity promotes coexistence in a mutualist/exploiter community. {\bf Workshop 5: Coevolution and the ecological structure of plant-insect communities} at the Mathematical Biosciences Institute in Columbus, Ohio.
\end{itemize}

\section*{Professional Affiliations}
\begin{itemize}
%\item Ecological Society of America
%\item Society for Conservation Biology %American Society of Naturalists%, European Society for Evolutionary Biology
\item British Ecological Society
%\item European Society for Evolutionary Biology
\item Association for the Study of Animal Behaviour
\item International Society for the History, Philosophy, and Social Studies of Biology
%\item The Genetics Society
\end{itemize}


%\section*{References}

%\begin{minipage}{0.5\linewidth}

%\begin{itemize}
%\item {\bf Prof. Jane Reid} \\
%School of Biological Sciences \\
%Zoology Building \\
%University of Aberdeen \\
%Aberdeen, UK, AB24 2TZ \\
%Phone: +44 01224 274224 \\
%Email: jane.reid@abdn.ac.uk \\

%\item {\bf Prof. Nils Bunnefeld} \\
%Department Biological and \\ 
%Environmental Sciences \\
%University of Stirling \\
%3B164, Stirling, UK, FK9 4LA \\
%Phone: +44 1786 467804 \\
%Email: nils.bunnefeld@stir.ac.uk \\

%\end{itemize}
%\end{minipage}
%\begin{minipage}{0.5\linewidth}

%\begin{itemize}

%\newline
%\item {\bf Prof. John Nason} \\
%Department of Ecology, Evolution, \\ and Organismal Biology \\
%339 Bessey Hall \\
%Iowa State University \\
%Ames, Iowa, USA 50011 \\
%Phone: 1-515-294-2268 \\
%Email: jnason@iastate.edu

%\item {\bf Prof. Justin Travis} \\
%School of Biological Sciences \\
%Zoology Building \\
%University of Aberdeen \\
%Aberdeen, UK, AB24 2TZ \\
%Phone: +44 01224 274483 \\
%Email: justin.travis@abdn.ac.uk \\

%\item {\bf Prof. Brent Danielson} \\
%Department of Ecology, Evolution, \\ and Organismal Biology \\
%253 Bessey Hall \\
%Iowa State University \\
%Ames, Iowa, USA 50011 \\
%Phone: 1-515-294-5248 \\
%Email: brentd@iastate.edu


%\end{itemize}

%\end{minipage}

%\bigskip

% Footer
%\begin{center}
%  \begin{footnotesize}
%    Last updated: \today \\
%    \href{\footerlink}{\texttt{\footerlink}}
%  \end{footnotesize}
%\end{center}

\end{document}
